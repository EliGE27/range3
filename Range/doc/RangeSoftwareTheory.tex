%\documentclass{article}
\documentclass[a4paper,10pt]{book}
\usepackage{amsmath}
\usepackage{times}
\usepackage{makeidx}
\usepackage{hyperref}
\hypersetup{
    colorlinks,
    citecolor=black,
    filecolor=black,
    linkcolor=black,
    urlcolor=black
}

\title{Range Software Theory}
\author{Tom\'{a}\v{s} \v{S}oltys\\
  Range Software\\
  \texttt{tomas.soltys@range-software.com}}
\date{\today}

\makeindex

\begin{document}

\frontmatter

% TITLE PAGE

\thispagestyle{empty}
\maketitle

% TABLE OF CONTENTS

\setcounter{tocdepth}{2}
\tableofcontents

\mainmatter

% INTRODUCTION

\chapter{Introduction}
Some introduction ...

% GOVERNING EQUATIONS

\chapter{Governing equations}

\section{Heat transfer}

\subsection{Conduction}

    \begin{equation}
    \rho{c}\frac{\partial{T}}{\partial{t}}=-k\nabla{T}
    \end{equation}

    \begin{equation}
    \rho{c}\sum_{j=1}^{nf}\frac{\partial{T_j}}{\partial{t}}=-k\sum_{j=1}^{nf}\nabla{T_j}
    \end{equation}

\subsection{Convection}

    \begin{equation}
    q_h=hA\left(T_h-T\right)
    \end{equation}

    \begin{equation}
    q_h^i=h\sum_{j=1}^{nf}A_j\left(T_h-T_j\right)
    \end{equation}

\subsection{Radiation}

    \begin{equation}
    \sum_{j=1}^{np}
        \left(
            \frac{\delta_{ij}}{\varepsilon_j}-F_{ij}\frac{1-\varepsilon_j}{\varepsilon_j}
        \right)q_r^j=
    \sum_{j=1}^{np}\left(\delta_{ij}-F_{ij}\right)\sigma{T_j^4}-
    \left(1-\sum_{j=1}^{np}F_{ij}\right)\sigma{T_a^4}
    \end{equation}

    \begin{equation}
    A_{ij}=\frac{\delta_{ij}}{\varepsilon_j}-F_{ij}\frac{1-\varepsilon_j}{\varepsilon_j}
    \end{equation}

    \begin{equation}
    B_{ij}=\left(\delta_{ij}-F_{ij}\right)\sigma
    \end{equation}

    \begin{equation}
    q_{ra}^i=\left(1-\sum_{j=1}^{np}F_{ij}\right)\sigma{T_a^4}
    \end{equation}

    \begin{equation}
    \sum_{j=1}^{np}A_{ij}q_r^j=\sum_{j=1}^{np}B_{ij}T_j^4-q_{ra}^i
    \end{equation}

\subsection{Conservation of energy}

    \begin{equation}
    \rho{c}\frac{\partial{T}}{\partial{t}}+k\nabla{T}=q_h+q_r+q
    \end{equation}

\subsubsection{Conduction-Convection equation}

    \begin{equation}
    \rho{c}\sum_{j=1}^{nf}\frac{\partial{T_j}}{\partial{t}}+k\sum_{j=1}^{nf}\nabla{T_j}=
    h\sum_{j=1}^{nf}A_j\left(T_h-T_j\right)+q_r^i+q^i
    \end{equation}

    \begin{equation} \label{eq:condconv}
    \rho{c}\sum_{j=1}^{nf}\frac{\partial{T_j}}{\partial{t}}+
    k\sum_{j=1}^{nf}\nabla{T_j}+
    h\sum_{j=1}^{nf}A_jT_j=
    h\sum_{j=1}^{nf}A_jT_h+q_r^i+q^i\;;\;i\in\langle{1,nf}\rangle
    \end{equation}

\subsubsection{Radiation equation}

    \begin{equation}
    \rho{c}\sum_{j=1}^{np}B_{ij}T_j^4=h\sum_{j=1}^{np}A_{ij}
    \end{equation}

    \begin{equation}
    q_r^j=q^j+q_h^j
    \end{equation}

    \begin{equation}
    q_r^j=q^j+h\left(T_h-T_j\right)
    \end{equation}

    \begin{equation}
    \sum_{j=1}^{np}B_{ij}T_j^4=
    \sum_{j=1}^{np}A_{ij}\left(q^j+h\left(T_h-T_j\right)\right)+q_{ra}^i
    \end{equation}

    \begin{equation} \label{eq:radtemp}
    \sum_{j=1}^{np}\left(B_{ij}T_j^4+A_{ij}hT_j\right)=
    \sum_{j=1}^{np}A_{ij}\left(q^j+hT_h\right)+q_{ra}^i\;;\;i\in\langle1,np\rangle
    \end{equation}

    \begin{equation} \label{eq:radflux}
    q_r^i=\sigma{T_i^4}-q^i-h\left(T_h-T_i\right)\;;\;i\in\langle1,np\rangle
    \end{equation}

\subsubsection{Coupling algorithm}

\begin{enumerate}
    \item Solve for temperature in equation ~\ref{eq:radtemp}
    \item Calculate the radiative heat flux ~\ref{eq:radflux}
    \item Solve equation ~\ref{eq:condconv}
\end{enumerate}

\subsection{Notation}

\begin{tabular}{|l|l|c|}
    \hline
    \multicolumn{3}{|c|}{Variables description} \\
    \hline
    $T$ & Temperature & $[K]$ \\
    $q$ & Heat flux & $[W\cdot{m^{-2}}]$ \\
    $\rho$ & Material density & $[kg\cdot{m^{-3}}]$ \\
    $c$ & Material heat capacity & $[J\cdot{kg^{-3}}\cdot{K^{-1}}]$ \\
    $k$ & Material thermal conductivity & $[W\cdot{m^{-1}}\cdot{K^{-1}}]$ \\
    $h$ & Convection coefficient & $[W\cdot{m^{-2}}\cdot{K^{-1}}]$ \\
    $\sigma$ & Stefan–Boltzmann constant $\sigma=5.670373\left(21\right)\times10^{-8}$ & $[W{\cdot}m^{-2}{\cdot}K^{-4}]$ \\
    $\varepsilon_{j}$ & Emissivity & $NA$ \\
    $F_{ij}$ & View factor $F_{ij}=\langle0;1\rangle$ & $NA$ \\
    $\delta_{ij}$ & Dirac delta function & $NA$ \\
    \hline
\end{tabular}

\section{Electromagnetism}

\subsection{Electrostatics}

\subsubsection{Electrical potential}
    \begin{equation}
    \nabla^2\phi=-\frac{\rho}{\varepsilon_r\varepsilon_0}
    \end{equation}

\subsubsection{Electric field}
    \begin{equation}
    \mathbf{E}=-\nabla\phi
    \end{equation}

\subsubsection{Current density}
    \begin{equation}
    \mathbf{J}=\sigma\mathbf{E}
    \end{equation}

\subsubsection{Joule heating}
    \begin{equation}
    Q=\sigma\int\mathbf{E}\cdot\mathbf{E}\mathrm{d}V
    \end{equation}

\subsubsection{Energy per unit volume of the electric field}
    \begin{equation}
    u^e=\frac{1}{2}\varepsilon_0\mathbf{E}^2
    \end{equation}

\subsection{Magnetostatics}

\subsubsection{Curl identity operation}
    \begin{equation}
    \nabla\times\left(\nabla\times\mathbf{F}\right)=\nabla\left(\nabla\cdot\mathbf{F}\right)-\nabla^2\mathbf{F}
    \end{equation}

    \begin{equation}
    \nabla{f}=\left[\begin{array}{c}{\frac{\partial{f}}{\partial{x}}}\\{\frac{\partial{f}}{\partial{y}}}\\{\frac{\partial{f}}{\partial{z}}}\end{array}\right]
    \end{equation}

    \begin{equation}
    \nabla^2{f}=\frac{\partial^2{f}}{\partial{x^2}} + \frac{\partial^2{f}}{\partial{y^2}} + \frac{\partial^2{f}}{\partial{z^2}}
    \end{equation}

    \begin{equation}
    \nabla\cdot\mathbf{F}=\left(\frac{\partial{F_x}}{\partial{x}}+\frac{\partial{F_y}}{\partial{y}}+\frac{\partial{F_z}}{\partial{z}}\right)
    \end{equation}

    \begin{equation}
    \nabla\left(\nabla\cdot\mathbf{F}\right)=
        \left[
            \begin{array}{c}
                \frac{\partial}{\partial{x}}\left(\frac{\partial{F_x}}{\partial{x}}+\frac{\partial{F_y}}{\partial{y}}+\frac{\partial{F_z}}{\partial{z}}\right) \\
                \frac{\partial}{\partial{y}}\left(\frac{\partial{F_x}}{\partial{x}}+\frac{\partial{F_y}}{\partial{y}}+\frac{\partial{F_z}}{\partial{z}}\right) \\
                \frac{\partial}{\partial{z}}\left(\frac{\partial{F_x}}{\partial{x}}+\frac{\partial{F_y}}{\partial{y}}+\frac{\partial{F_z}}{\partial{z}}\right)
            \end{array}
        \right]
    \end{equation}

    \begin{equation}
    \nabla^2\mathbf{F}=
        \left[
            \begin{array}{c}
                \frac{\partial^2{F_x}}{\partial{x^2}}+\frac{\partial^2{F_x}}{\partial{y^2}}+\frac{\partial^2{F_x}}{\partial{z^2}} \\
                \frac{\partial^2{F_y}}{\partial{x^2}}+\frac{\partial^2{F_y}}{\partial{y^2}}+\frac{\partial^2{F_y}}{\partial{z^2}} \\
                \frac{\partial^2{F_z}}{\partial{x^2}}+\frac{\partial^2{F_z}}{\partial{y^2}}+\frac{\partial^2{F_z}}{\partial{z^2}}
            \end{array}
        \right]
    \end{equation}

    \begin{equation}
    \nabla\times\mathbf{F}=
        \left[
            \begin{array}{c c c}
                0 & -\frac{\partial}{\partial{z}} & \frac{\partial}{\partial{y}} \\
                \frac{\partial}{\partial{z}} & 0 & -\frac{\partial}{\partial{x}} \\
                -\frac{\partial}{\partial{y}} & \frac{\partial}{\partial{x}} & 0
            \end{array}
        \right]
        \left[
            \begin{array}{c}
                F_x \\ F_y \\ F_z
            \end{array}
        \right]
    \end{equation}

\subsubsection{Gauss's law for magnetism}

    \begin{equation}
    \nabla\cdot\mathbf{B} = 0
    \end{equation}

\subsubsection{Ampere's circuital law}

    \begin{equation}
    \nabla\times\mathbf{B} = \mu_0\mathbf{J}+\underbrace{\mu_0\varepsilon_0\frac{\partial\mathbf{E}}{\partial{t}}}_{\textrm\footnotesize{if steady state} \Rightarrow 0}
    \end{equation}

    \begin{equation}
    \nabla\times\left(\nabla\times\mathbf{B}\right)=\nabla\times\left(\mu_0\mathbf{J}\right)
    \end{equation}

    \begin{equation}
    \nabla\times\left(\nabla\times\mathbf{B}\right)=\nabla\left(\underbrace{\nabla\cdot\mathbf{B}}_{0}\right)-\nabla^2\mathbf{B}
    \end{equation}

    \begin{equation}
    -\nabla^2\mathbf{B}=\nabla\times\left(\mu_0\mathbf{J}\right)
    \end{equation}

\subsection{Notation}

\begin{tabular}{|l|p{6.7cm}|c|}
    \hline
    \multicolumn{3}{|c|}{Variables description} \\
    \hline
    $\phi$ & electric potential & $[J\cdot{C^{-1}}] \vee [V]$ \\
    $\mathbf E$ & electric field & $[N\cdot{C^{-1}}] \vee [V\cdot{m^{-1}}]$ \\
    $\mathbf J$ & current density & $[A\cdot{m^{-2}}]$ \\
    $\mathbf B$ & magnetic field & $[T]$ \\
    $Q$ & Joule heat & $[W]$ \\
    $u_e$ & energy per unit volume of the electric field & $[J\cdot{m^{-3}}]$ \\
    $\rho$ & charge density & $[C\cdot{m^{-3}}]$ \\
    $\varepsilon_r$ & relative permittivity of the medium (dielectric constant) & $[C^2\cdot{N^{-1}m^{-2}})]$ \\
    $\varepsilon_0$ & vacuum permittivity $\varepsilon_0 = 8.854187817\times10^{-12}$ & $[C^2\cdot{N^{-1}m^{-2}})]$ \\
    $\sigma$ & electrical conductivity & $[S\cdot{m^{-1}}]$ \\
    $\mu_0$ & vacuum permeability $\mu_0 = 4\pi\times10^{-7}$ & $[V{\cdot}s{\cdot}A^{-1}{\cdot}m^{-1} \vee N{\cdot}A^{-2}]$ \\
    \hline
\end{tabular}

\section{Waves}

\subsection{General Wave}

    \begin{equation}
    \frac{\partial^2u}{\partial{t}^2}+d\frac{\partial{u}}{\partial{t}}=
    {c(u)}^2\left(\nabla^2u\right)
    \end{equation}

    For linear case $c$ is a fixed constant equal to the propagation speed of wave. For acoustic $c=343 \left[m{\cdot}s^-1\right]$ (speed of sound in dry air at $20 \left[C\right]$)

\subsubsection{Newmark-beta method}

    \begin{equation}
    \ddot{u}_{\beta} = \left(1-2\beta\right)\ddot{u}_{n} + 2 \beta\ddot{u}_{n+1} \qquad 0 \leq 2\beta \leq 1
    \end{equation}

    \begin{equation}
    \dot{u}_{n+1} = \dot{u}_{n} + \frac{\Delta t}{2} \left( \ddot{u}_{n} + \ddot{u}_{n+1} \right)
    \end{equation}

    \begin{equation}
    u_{n+1} = u_{n} + \Delta t \dot{u}_{n} + \frac{1-2\beta}{2} \Delta t^{2} \ddot{u}_{n} + \beta \Delta t^{2} \ddot{u}_{n+1}
    \end{equation}

    \begin{equation}
    \mathbf{A}\ddot{\mathbf{u}}_{n+1} + \mathbf{B}\dot{\mathbf{u}}_{n+1} + \mathbf{C}\mathbf{u}_{n+1} = \mathbf{F}_{n+1}
    \end{equation}

    \begin{equation}
    \begin{aligned}
    \mathbf{A}\ddot{\mathbf{u}}_{n+1}
    + \mathbf{B}\left( \dot{\mathbf{u}}_{n} + \frac{\Delta t}{2} \left( \ddot{\mathbf{u}}_{n} + \ddot{\mathbf{u}}_{n+1} \right) \right) \\
    + \mathbf{C}\left( \mathbf{u}_{n} + \Delta t \dot{\mathbf{u}}_{n} + \frac{1-2\beta}{2} \Delta t^{2} \ddot{\mathbf{u}}_{n} + \beta \Delta t^{2} \ddot{\mathbf{u}}_{n+1} \right)
    = \mathbf{F}_{n+1}
    \end{aligned}
    \end{equation}

\iffalse
    \begin{equation}
    \begin{aligned}
    \end{aligned}
    \end{equation}
\fi

\subsubsection{Discretized equation}

    \begin{equation}
    \begin{split}
    \left(
        \left(1+d\Delta t\right)\mathbf{M}-\Delta t^2\mathbf{K}
    \right)\mathbf{u}^{(n)}&=
    \Delta t^2\mathbf{f}^{(n)}\\&+\mathbf{M}
    \left(
        \left(2+d\Delta t\right)\mathbf{u}^{(n-1)}-\mathbf{u}^{(n-2)}
    \right)
    \end{split}
    \end{equation}
    \begin{equation}
    \begin{split}
    \left(
        \left(1+d\Delta t\right)\mathbf{M}-\alpha\Delta t^2\mathbf{K}
    \right)\mathbf{u}^{(n)}&=
    \Delta t^2\mathbf{f}^{(n)}\\&+\mathbf{M}
    \left(
        \left(2+d\Delta t\right)\mathbf{u}^{(n-1)}-\mathbf{u}^{(n-2)}
    \right)\\
    &-\left(
        \left(1-\alpha\right)\Delta t^2\mathbf{K}
    \right)\mathbf{u}^{(n-1)}
    \end{split}
    \end{equation}

\subsubsection{Absorbing boundary}

    This technique was developed by Higdon (1991).

    \begin{equation}
    u_{t+1}=-q_xu_{t+1,2}-q_tU_{t,1}-q_{tx}u_{t,2}
    \end{equation}

    Where:

    \begin{eqnarray*}
    q_x&=&\frac{b\left(B+V\right)-V}{\left(B+V\right)\left(1-b\right)}\\
    q_t&=&\frac{b\left(B+V\right)-B}{\left(B+V\right)\left(1-b\right)}\\
    q_{tx}&=&\frac{b}{b-1}\\
    b&=&0.4\\
    B&=&1\\
    V&=&v\frac{\Delta t}{\Delta x}\\
    v&-&\text{velocity of the wavefront normal to the boundary}
    \end{eqnarray*}

\subsection{Elastic Wave}

    \begin{equation}
    \rho\frac{\partial^2\mathbf{u}}{\partial{t}^2}=
    \mathbf{f}+\left(\lambda+2\mu\right)\nabla\left(\nabla\cdot\mathbf{u}\right)-
    \mu\nabla\times\left(\nabla\times\mathbf{u}\right)
    \end{equation}

\subsubsection{Lam\'{e} parameters}

    \begin{equation}
    \lambda=\frac{E\nu}{\left(1+\nu\right)\left(1-2\nu\right)}
    \end{equation}

    \begin{equation}
    \mu=\frac{E}{2\left(1+\nu\right)}
    \end{equation}

\subsection{Notation}

\begin{tabular}{|l|l|c|}
    \hline
    \multicolumn{3}{|c|}{Variables description} \\
    \hline
    $c$ & Propagation speed of wave & $[m\cdot{s^{-1}}]$ \\
    $d$ & Damping parameter & $[1\cdot{s^{-1}}]$ \\
    $\lambda$ & Lam\'{e}'s first parameter & $[GPa]$ \\
    $\mu$ & Lam\'{e}'s second parameter & $[GPa]$ \\
    $E$ & Young's modulus & $[GPa]$ \\
    $\nu$ & Poison's ratio & $NA$ \\
    $\rho$ & Density & $[kg\cdot{m^{-3}}]$ \\
    $\mathbf f$ & source function (driving force) & $NA$ \\
    $\mathbf u$ & wave displacement & $NA \vee [m]$ \\
    $t$ & time & $[s]$ \\
    \hline
\end{tabular}

\section{Incompressible newtonian fluids}

\subsection{Navier-Stokes equations of incompressible flows}
    Incompressible fluid flow is described by two equations. These are \textbf{momentum equation} ~\ref{eq:nsmomentum} and \textbf{incompressibility equation} ~\ref{eq:incompressibility}.
    \begin{equation}
    \label{eq:nsmomentum}
    \rho\left(\frac{\partial\mathbf{u}}{\partial{t}}
    +\mathbf{u}\cdot\nabla\mathbf{u}-\mathbf{f}\right)
    -\nabla\cdot\boldsymbol{\sigma}=0
    \end{equation}
    \begin{equation}
    \label{eq:incompressibility}
    \nabla\cdot\mathbf{u}=0
    \end{equation}
    For Newtonian fluids stress tenzor $\boldsymbol\sigma$ can be expressed with:
    \begin{equation}
    \boldsymbol{\sigma}=-p\mathbf{I}+2\mu\boldsymbol{\varepsilon}
    \end{equation}
    Where $\mathbf{I}$ is an identity matrix $\boldsymbol\varepsilon$ is a strain-rate tensor and $\mu$ is dynamic viscosity.
    \begin{equation}
    \boldsymbol{\varepsilon}=\frac{1}{2}
    \left(
        \nabla\mathbf{u}+\left(\nabla\mathbf{u}\right)^T
    \right)
    \end{equation}
    \begin{equation}
    \mu=\rho\nu
    \end{equation}
    \begin{equation}
    \nabla\cdot\left(2\mu\boldsymbol{\varepsilon}\right)=
    \mu\nabla\cdot\left(2\boldsymbol{\varepsilon}\right)
    \end{equation}
    \begin{equation}
    \begin{split}
    \nabla\cdot\left(2\boldsymbol{\varepsilon}\right)
    &=\nabla\cdot\left(\nabla\mathbf{u}+\left(\nabla\mathbf{u}\right)^T\right) \\
    &=\nabla\cdot\left(\nabla\mathbf{u}\right)+\nabla\cdot\left(\nabla\mathbf{u}\right)^T \\
    &=\nabla^2\mathbf{u}+\nabla\underbrace{\left(\nabla\cdot\mathbf{u}\right)}_{0} \\
    &=\nabla^2\mathbf{u}
    \end{split}
    \end{equation}
    Where $\nabla^2$ is Laplace operator.
    \begin{equation}
    \nabla^2=\frac{\partial^2}{\partial{x_1^2}}
            +\frac{\partial^2}{\partial{x_2^2}}
            +\frac{\partial^2}{\partial{x_3^2}}
    \end{equation}
    \begin{equation}
    \label{eq:divstress}
    \nabla\cdot\boldsymbol\sigma=-\nabla p+\mu\nabla^2\mathbf{u}
    \end{equation}
    Applying equation \ref{eq:divstress} to \ref{eq:nsmomentum} results in following set of the \textbf{Navier-Stokes equations of incompressible flows}.
    \begin{equation}
    \label{eq:gnsmomentum}
    \rho\left(\frac{\partial\mathbf{u}}{\partial{t}}
    +\mathbf{u}\cdot\nabla\mathbf{u}-\mathbf{f}\right)
    +\nabla p-\mu\nabla^2\mathbf{u}=0
    \end{equation}
    \begin{equation}
    \label{eq:gincompressibility}
    \nabla\cdot\mathbf{u}=0
    \end{equation}
    Equations \ref{eq:gnsmomentum} and \ref{eq:gincompressibility} written in integral form:
    \begin{equation}
    \int\limits_\Omega\mathbf{w}\cdot\rho
    \left(
        \frac{\partial\mathbf{u}}{\partial{t}}+\mathbf{u}\cdot\nabla\mathbf{u}-\mathbf{f}
    \right)d\Omega
    +\int\limits_\Omega\boldsymbol\varepsilon\left(\mathbf{w}\right):\boldsymbol\sigma{d}\Omega
    +\int\limits_\Omega{q}\left(\nabla\cdot\mathbf{u}\right)d\Omega
    =\int\limits_\Gamma\mathbf{w}\cdot\mathbf{h}d\Gamma
    \end{equation}

\subsection{Non-dimenzionalization}

    Non-dimensionalize the equation (steady-state version with $\mathbf{f}=0$)
    \begin{eqnarray}
    &\mathbf{u}=\mathbf{u}^*U \\
    &\nabla=\nabla^*\frac{1}{L} \\
    &\nabla^2=\left(\nabla^*\right)\frac{1}{L}
    \end{eqnarray}

\subsection{Reynolds number}

    \begin{equation}
    R_e=\frac{UL}{\nu}
    \end{equation}
    If $R_e$ goes to zero, then we are solving stokes flow, if it goes to infinity, then we are "approaching" inviscid flow.

\subsubsection{Solid surface}

    If viscosity not equals to and it is viscous flow and therefore $\mathbf{u}=\mathbf{0}$.\\
    If unviscous boundary is assumed than only normal component of velocity is equal to zero.

\subsubsection{Free surface}

    \begin{equation}
    \mathbf{n}\cdot\boldsymbol{\sigma}=-p_{atm}\mathbf{n}
    \end{equation}
    If pressure is scaled so that $p_{atm}=0$, then
    \begin{equation}
    \mathbf{n}\cdot\boldsymbol{\sigma}=\mathbf{0}
    \end{equation}
    If there are two liquids, then surface is not a boundary but interface. \\
    Normal velocity of both liquids on the interface must be equal.

\subsection{External boundaries}

    External boundaries are assumed to be sufficiently far from the object, so that we can approximate the boundary conditions with free-stream conditions.

\subsubsection{Free-stream conditions}

    \begin{equation}
    \mathbf{u}=\mathbf{u}_\infty
    \end{equation}
    or
    \begin{equation}
    \begin{split}
    \boldsymbol{\sigma}&=\boldsymbol{\sigma}_\infty \\
                       &=-p_\infty\mathbf{I}+2\mu\boldsymbol\varepsilon\left(\mathbf{u}_\infty\right) \\
                       &=-p_\infty\mathbf{I}+\mu\left(\nabla\mathbf{u}_\infty+\left(\nabla\mathbf{u}_\infty\right)^T\right)
    \end{split}
    \end{equation}
    More general:
    $u_i=\left(u_infty\right)_i$ or $\left(\mathbf{n}\cdot\boldsymbol{\sigma}\right)_i$
    In most cases:
    \begin{equation*}
    \mathbf{u}_\infty=(U,0,0)
    \Rightarrow\nabla\mathbf{u}_\infty=\mathbf{0}
    \Rightarrow\boldsymbol{\sigma}_\infty=-p_\infty\mathbf{I}
    \Rightarrow\mathbf{n}\cdot\boldsymbol{\sigma}_\infty=-\mathbf{n}p_\infty
    \end{equation*}

\subsection{Boundary conditions in general}

    \begin{equation}
    \mathbf{u}=\mathbf{q}
    \end{equation}
    \begin{equation}
    \mathbf{n}\cdot\boldsymbol{\sigma}=\mathbf{h}
    \end{equation}
    This means:
    \begin{equation*}
    u_i=g_i\;\;\text{or}\;\;(\mathbf{n}\cdot\boldsymbol{\sigma})_i=h_i
    \end{equation*}
    If assuming solution in liquid, than hydrostatic pressure should be encounted. \\
    Substitution can be used:
    \begin{equation}
    p^*=p-p_h(z)
    \end{equation}

\subsection{Spatial discretization}

    Time-discretization for the case of time-lagging update fot $\tau$,
    and simultaneous update for $\left(\mathbf{u}\cdot\nabla\right)\mathbf{w}$:
    \begin{equation}
    \begin{split}
    \mathbf{M}\frac{\mathbf{U}_{n+1}-\mathbf{U}_n}{\Delta{t}}
    +\mathbf{M}_{\stackrel{\sim}{c}}\left(\mathbf{U}_{n+\alpha}\right)
    \frac{\mathbf{U}_{n+1}-\mathbf{U}_n}{\Delta{t}} & \\
    +\mathbf{N}\left(\mathbf{U}_{n+\alpha}\right)
    +\mathbf{N}_{\stackrel{\sim}{k}}\left(\mathbf{U}_{n+\alpha}\right)
    +\mathbf{K}_e\mathbf{U}_{n+1}+\mathbf{K}\mathbf{U}_{n+\alpha} & \\
    -\mathbf{G}\mathbf{P}_{n+1}-\mathbf{G}_{\stackrel{\sim}{\gamma}}\left(\mathbf{U}_{n+\alpha}\right)\mathbf{P}_{n+1}&=\left(\mathbf{F}+\mathbf{F}_s\right)_{n+\alpha}
    \end{split}
    \end{equation}
    \begin{equation}
    \mathbf{H}_\beta\frac{\mathbf{U}_{n+1}-\mathbf{U}_n}{\Delta{t}}
    +\mathbf{N}_\gamma\left(\mathbf{U}_{n+\alpha}\right)
    +\mathbf{G}^T\mathbf{U}_{n+1}+\mathbf{L}_\theta\mathbf{P}_{n+1}
    =\left(\mathbf{E}+\mathbf{E}_s\right)_{n+\alpha}
    \end{equation}
    Where:
    \begin{eqnarray*}
    & \mathbf{M}\approx\mathbf{m} & \\
    & \mathbf{M}_{\stackrel{\sim}{c}}\approx\mathbf{\stackrel{\sim}{c}} & \\
    & \mathbf{N}\approx\mathbf{w}\cdot\rho\left(\mathbf{u}\cdot\nabla\right)\mathbf{u}\approx\mathbf{c} & \\
    & \mathbf{N}_{\stackrel{\sim}{k}}\approx\tau_{SUPG}\left(\mathbf{u}\cdot\nabla\right)\mathbf{w}\cdot\rho\left(\mathbf{u}\cdot\nabla\right)\mathbf{u}\approx\mathbf{\stackrel{\sim}{k}} & \\
    & \mathbf{K}_e\approx\tau_{LSIC}\left(\nabla\cdot\mathbf{w}\right)\rho\left(\nabla\cdot\left(\mathbf{u}\right)\right)\approx\mathbf{e} & \\
    & \mathbf{K}\approx\mathbf{k} & \\
    & \mathbf{G}\approx\mathbf{g} & \\
    & \mathbf{G}_{\stackrel{\sim}{\gamma}}\approx\tau_{SUPG}\left(\mathbf{u}\cdot\nabla\right)\mathbf{w}\cdot\nabla{p}\approx\boldsymbol{\stackrel{\sim}{\gamma}} & \\
    & \mathbf{H}_{\stackrel{\sim}{\beta}}\approx\beta & \\
    & \mathbf{N}_\gamma\approx\tau_{PSPG}\nabla{q}\cdot\left(\mathbf{u}\cdot\nabla\right)\mathbf{u}\approx\boldsymbol{\gamma} & \\
    & \mathbf{L}_\theta\approx\tau_{PSPG}\nabla{q}\cdot\nabla{p}\frac{1}{\rho}\approx\boldsymbol\theta
    \end{eqnarray*}

\subsubsection{Unsteady formulation}

\subsubsection{Steady-state formulation}

\subsection{Notation}

\begin{tabular}{|l|l|c|}
    \hline
    \multicolumn{3}{|c|}{Variables description} \\
    \hline
    $\rho$ & Density & $[kg\cdot{m^{-3}}]$ \\
    $\mu$ & Dynamic viscosity & $[N \cdot s \cdot m^{-2}] \vee [kg \cdot m^{-1} \cdot s^{-1}]$ \\
    $\nu$ & Kinematic viscosity & $[m^2 \cdot s^{-1}]$ \\
    $t$ & time & $[s]$ \\
    \hline
\end{tabular}

\backmatter

\begin{thebibliography}{9}
    \bibitem{RelativeStaticPermittivity}
        http://en.wikipedia.org/wiki/Relative\_static\_permittivity
    \bibitem{ElectricalConductivity}
        http://en.wikipedia.org/wiki/Electrical\_conductivity
    \bibitem{CurlMathematics}
        http://en.wikipedia.org/wiki/Curl\_\%28mathematics\%29
    \bibitem{MaxwellEquations}
        http://en.wikipedia.org/wiki/Maxwell\%27s\_equations
    \bibitem{WaveEquation}
        http://en.wikipedia.org/wiki/Wave\_equation
    \bibitem{AcousticWaveEquation}
        http://en.wikipedia.org/wiki/Acoustic\_wave\_equation
    \bibitem{LamesParameters}
        http://en.wikipedia.org/wiki/Lam\%C3\%A9\_parameters
\end{thebibliography}

\printindex

\end{document}
